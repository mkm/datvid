\documentclass{beamer}
\usepackage[utf8]{inputenc}
\usetheme{Warsaw}
\title[Datalogi i uddannelsen]{Datalogi i uddannelsen}
\author{Mikkel Kragh, Jannik Gram, Rune Abrahamsson, Rasmus Abrahamsen}
\institute{DIKU}
\date{\today}
\begin{document}

\begin{frame}
\titlepage
\end{frame}


\begin{frame}{Problemet}
Nu om dage bruger folk mere tid foran computeren, hvilket i stigende grad gør den uundgåelig.

Folk er uvidende og generelt meget dårlige til at benytte sig af computere.

De benytter sig af værktøjer, som ikke er bygget efter deres behov fordi de ikke ved bedre.

F.eks. gruppearbejde hvor man er fælles om et Word-dokument, som er delt mellem medlemmerne vha. email.

\end{frame}

\begin{frame}{Datalogi i folkeskolen}

Der skal indføres datalogi i folkeskolen

\end{frame}

\begin{frame}{Datalogi i gymnasiet}

og i gymnasiet.

\end{frame}

\begin{frame}{Perspektivering}

Om nogle år bliver det nice.

\end{frame}

\begin{frame}{TAK}

\end{frame}


\end{document}
