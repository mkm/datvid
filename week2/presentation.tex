\documentclass{article}
\usepackage[utf8]{inputenc}
\usepackage{graphicx}
\usepackage{fancyhdr}
\usepackage{verbatim}
\pagestyle{fancyplain}
\author{Mikkel Kragh Mathiesen, Jannik Gram, Rune \& Rasmus Abrahams{\tt (son|en)}}
\title{}
\date{\today}
\lhead{Mikkel, Jannik, Rune \& Rasmus}
\rhead{\today}
\begin{document}

\maketitle

\section{Analyse af modellerings- og simuleringsprocesser}
\subsection*{(1.a)}
Artiklen præsenterer problemstillinger fra den virkelige verden omhandlende udfordringen med at afgøre hvordan de forskellige led på en robot eller et menneske skal bevæge sig for at opnå forskellige mål. Dette hører ind under boksen ``Real World''. Med dette i mente opstilles der en model under en proces, der retteligt kan betegnes som idealisering. Den resulterende model er matematisk af karakter og hører ind under boksen af samme navn. Til sidst ligger den matematiske model grund til udformningen af en algoritme, der hører ind under boksen ``diskret model''. BLAbla diskretisering...

Real World
idealisation
math model
discretisation
discrete model

verification
simulation
results
validation

\section{Boldsimulator}

\section{Kantfinder}

\end{document}
