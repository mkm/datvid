\documentclass{article}
\usepackage[utf8]{inputenc}
\usepackage{graphicx}
\usepackage{fancyhdr}
\usepackage{verbatim}
\usepackage[danish]{babel}
\pagestyle{fancyplain}
\author{Mikkel K. Mathiesen, Jannik Gram, Rune \& Rasmus Abrahams{\tt (so|e)}n}
\title{}
\date{\today}
\lhead{Mikkel, Jannik, Rune \& Rasmus}
\rhead{\today}
\begin{document}
\maketitle

\section{Må salonen registrere kundernes personnumre?}
Nej. Det strider imod dataloven. Salonen må ikke registrere kundernes personnumre. Der står i persondataloven at det vil kræve at der enten var en lov eller at man har personens udtrykkelige samtykke samt et sagligt formål. Da ingen af disse betingelser er opfyldte, er det ikke tilladt for salonen at registrere kundernes personnumre.
Der er heller ingen lov om at frisører skal registrere personnumre. Det der med mulighed ind.


\section{Kan (må) salonens hjemmeside vise et fotografi af en nyklippet prins Felix?}
Da et billede af prins Felix er identificerbart er det ikke tilladt uden prisens samtykke.
Man må gå ud fra at et billede af F er identificerbart.
Vi antager at frisøren har taget billedet.

\section{Må Lis Hansen registrere kunder med hovedbundsproblemer og hårtab?}
Nej. Det må antages at hovedbundsproblemer og hårtab hører under helbredsproblemer. Tænk for eksempel på en person, der behandles med kemoterapi og som derfor lider af hårtab.
Helbredsproblemer hører til kategorien følsomme oplysninger og må derfor ikke behandles medmindre kunden giver sit udtrykkelige samtykke.
Man kunne måske sige at det var et patientforhold, men det er det ikke. Det er et kundeforhold.
Så Lis Hansen må ikke registrere kunder med hovedbundsproblemer og hårtab.

\section{Må Skovdoktoren få adgang til salonens kunderegister?}
Nej. Man må ikke dele personoplysninger med andre uden udtrykkeligt samtykke.
Ligeledes må informationer om kunder må ikke blive brugt til andre formål end de blev til at starte med.
%Må tyve få adgang til ulåste butikker? LOLOLOL du er en noob.

Hvis det var en videnskabelig undersøgelse som er for samfundets bedste og som er anmeldt til datatilsynet.

Skovdokterens undersøgelse er ikke til samfundets bedste, men rettere et kommercielt formål.

\section{Hvad kan Grethe Nielsen gøre og hvad har hun ret til?}
Grethe har ret til at få oplysning om hvorvidt salonen har oplysninger om hende og få indsigt i disse. Grethe har også ret til at få sine oplysninger rettet eller slettet hvis de er forkerte. Hvis salonen ikke svarer tilfredsstillende på anmodningen inden for fire uger har Grethe ret til at klage til datatilsynet.

\section{Kan Niels Jensen gøres ansvarlig for eventuelle overtrædelser af persondataloven?}
Nej for Niels Jensen er ikke dataansvarlig for salonen. Han er databehandler og har derfor ikke noget ansvar.
Typisk ville Niels dog have en kontrakt med hende der damen, som indebærer at hvis der sker noget, skal han betale noget til hende.

\section{Beskriv forslag til sikkerhedsforanstaltninger i forbindelse med kunderegistret.}

Vi antager at frisørsalonens system ikke indeholder oplysninger som vi tidligere har argumenteret for ikke må behandles. Vi kender ikke meget til it-systemet så vi ved umiddelbart ikke hvad vi har med at gøre. Idet systemet ligger tilgængeligt på Internettet, er det nødvendigt at lave et login-system så kun den dataansvarlige har adgang som administrator og kunder har adgang til deres egne data. Dernæst er det vigtigt at sikre serveren så indbrud (hacking) ikke kan forekomme. Vigtige data såsom passwords skal hashes eller krypteres.

\end{document}
