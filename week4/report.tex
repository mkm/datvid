\documentclass{article}
\usepackage[utf8]{inputenc}
\usepackage{graphicx}
\usepackage{fancyhdr}
\usepackage{verbatim}
\usepackage[danish]{babel}
\pagestyle{fancyplain}
\author{Mikkel K. Mathiesen, Jannik Gram, Rune \& Rasmus Abrahams{\tt (so|e)}n}
\title{}
\date{\today}
\lhead{Mikkel, Jannik, Rune \& Rasmus}
\rhead{\today}
\begin{document}
\maketitle

\section{Må salonen registrere kundernes personnumre?}
Det strider umiddelbart mod persondataloven for den private sektor at registrere persomnumre, så salonen må ikke registrere kundernes personnumre. En sådan tilladelse ville kræve, at der enten var en anden lov, der gav anledning til eller krævede et særtilfælde, eller at man først sikrede sig kundens udtrykkelige samtykke -- og samtidig havde et sagligt formål med hvilket at bruge denne information. Da ingen af disse betingelser her er opfyldte, er det altså ikke tilladt for salonen at registrere kundernes personnumre. Det antages for eksempel, at det ikke er gældende lov, at frisører kan eller skal registrere kunders personnumre.

\section{Kan salonens hjemmeside vise et fotografi af en nyklippet prins Felix?}
Et fotografi må givetvis regnes for en identificerbar oplysning og kan således ikke behandles uden prinsens udtrykkelige samtykke. Her må vi antage, at frisøren har taget billedet og således ikke har fundet det andetsteds. Prinsen mister sine rettigheder jævnfør persondataloven, hvis fotografiet i forvejen er publiceret eller på anden vis offentligt tilgængeligt. (Andre love, navnlig loven om ophavsret, kan muligvis spille ind og forhindre brugen af billedet på salonens hjemmeside.)

\section{Må Lis Hansen registrere kunder med hovedbundsproblemer og hårtab?}
Idet ``hovedbundsproblemer og hårtab'' må høre under helbredsproblemer, er der tale om kategorien følsomme oplysninger. Under normale forhold, altså for en privat virksomheden eller organisation med et kunderegister, må disse informationer ikke behandles uden kundernes udtrykkelige samtykke. Som eksempel på, at disse helbredsproblemer kan anses som følsomme oplysninger, kan man tænke på en person, der behandles med kemoterapi, og som derfor lider af hårtab.

Man kunne måske argumentere for, at der er tale om et læge-patient-forhold, men formålet med databehandlingen er snarere kommercielt, og kunderne netop kunder. Lis Hansen må altså ikke registrere kunder med hovedbundsproblemer og hårtab.

\section{Må Skovdoktoren få adgang til salonens kunderegister?}
Skovdoktoren må ikke få adgang til salonens kunderegister; det er ikke tilladt for salonen at dele personoplysninger med andre uden udtrykkeligt samtykke. Ligeledes må personoplysninger ikke (senere) bruges til andre formål end de, personen oprindeligt gav sit samtykke til. En mulig undtagelse er, hvis der er tale om en videnskabelig undersøgelse, som er for samfundets bedste og som er anmeldt til datatilsynet. Skovdokterens undersøgelse er imidlertid kommercielt orienteret og kan næppe regnes for at være til samfundets bedste.

\section{Hvad kan Grethe Nielsen gøre og hvad har hun ret til?}
Grethe har ret til at få oplyst, hvilke oplysninger salonen eventuelt har registreret om hende. Grethe har videre krav på at få sine oplysninger rettet eller slettet, hvis de er forkerte, og har altid mulighed for at fratrække sit samtykke og få informationerne tilintetgjort. Hvis salonen ikke svarer tilfredsstillende på sådanne anmodninger inden for en frist på fire uger, har Grethe ret til at klage til datatilsynet.

\section{Kan Niels Jensen gøres ansvarlig for eventuelle overtrædelser af persondataloven?}
Niels Jensen er ikke dataansvarlig for salonen og kan derfor ikke gøres ansvarlig for eventuelle overtrædelser af persondataloven. Han er blot databehandler i denne sammenhæng og har ikke noget ansvar, medmindre --- som det typiske ville være tilfældet --- 

Typisk ville Niels dog have en kontrakt med hende der damen, som indebærer at hvis der sker noget, skal han betale noget til hende.

\section{Beskriv forslag til sikkerhedsforanstaltninger i forbindelse med kunderegistret.}

Vi antager at frisørsalonens system ikke indeholder oplysninger som vi tidligere har argumenteret for ikke må behandles. Vi kender ikke meget til it-systemet så vi ved umiddelbart ikke hvad vi har med at gøre. Idet systemet ligger tilgængeligt på Internettet, er det nødvendigt at lave et login-system så kun den dataansvarlige har adgang som administrator og kunder har adgang til deres egne data. Dernæst er det vigtigt at sikre serveren så indbrud (hacking) ikke kan forekomme. Vigtige data såsom passwords skal hashes eller krypteres.

\end{document}
