\documentclass{article}
\usepackage[utf8]{inputenc}
\usepackage{graphicx}
\usepackage{fancyhdr}
\usepackage{verbatim}
\usepackage[danish]{babel}
\pagestyle{fancyplain}
\author{Mikkel K. Mathiesen, Jannik Gram, Rune \& Rasmus Abrahams{\tt (son|en)}}
\title{}
\date{\today}
\lhead{Mikkel, Jannik, Rune \& Rasmus}
\rhead{\today}
\begin{document}
\maketitle
\section{Turings computermodel}
%Kun digitale maskiner i testen fordi de kan bevises at være universelle turing-maskiner.
%Altså kan en sådan digatal maskine simulere enhver discrete state maskine. Enhver digital maskine simulere enhver discrete state machine.

%(han tillader digitale maskiner fordi de er begrænset til at simulere diskrete maskiner)

%Forskellen: Digitale maskiner i teorien ikke kan være diskrete men i praksis (ikke er diskrete, eller ikke kan være, ikke er) de er ikke diskrete men kan i praksis ses som at være det. fordi man kan forudsige deres tilstand selvom en elektronspænding ikke er præcis så den regner rigtigt med det.
%Den er i sig selv en diskret tilstandsmaskine men samtidig … ideen er at i en virkelig verden er en digital maskine ikke en  diskret tilstandsmaskine men det kan abstraheres væk…

Turing tillader kun digitale maskiner (``digital computers'') at deltage i testen. En digital maskine er i den sammenhæng defineret som en maskine, der består af tre komponenter: ``store'', ``executive unit'' og ``control'', hvad der omtrentligt svarer til henholdsvis datalager, processor og kodelager i vor tids computere (dog er datalager og kodelager den samme komponent i von Neumann-arkitekturen).

Til sammenligning definerer Turing en diskret tilstandsmaskine (``discrete state machine'') som en maskine, der bevæger sig mellem diskrete tilstande. Det skal ses i modsætning til en kontinuer maskine, hvor der mellem to givne tilstande findes uendelig mange mellemliggende tilstande. Strengt taget findes sådanne diskrete maskiner ikke, så helt præcist er der tale om maskiner, der med rimelighed og uden tab af forudsigelighed, kan betragtes som diskrete i deres opførsel. Således er digitale maskiner en delmængde af diskrete maskiner.

Det skal dog bemærkes, at det reelt ikke ville ændre noget, hvis man tillod alle diskrete maskiner at deltage i testen. Idet en digital maskine er Turing-komplet, kan den simulere en hvilken som helst diskret maskine, og det vil derfor kun være et spørgsmål om ressourceforbrug, der adskiller en given fysisk diskret maskine og den tilsvarende emulerede diskrete maskine. 

\section{Computerhukommelsens betydning}
Ifølge Turings beregninger kan den menneskelige hjerne indeholde imellem $10^{10}$ og $10^{15}$ bits information, men da Turing-testen ikke kræver at en person kan se vil det ikke være nødvendigt med den mængde plads. Hvis man tager processering af visuelle data, genkendelse af ting og så videre væk, når han frem til at $10^9$ bits (hvilket svarer til en blind person) er nok til at klare Turing-testen med. Denne plads bliver brugt til en slags tabel af instruktioner som er det, en maskine, som skal klare Turing-testen består af. Med dette mener han at hvis hjernen kan indeholde $10^9$ bits hukommelse så skal en maskine, der direkte simulerer hjernen bruge samme mængde lager til sin instruktionstabel.

\section{Jeres prioritering af indvendingerne mod Turing-testen}
Overordnet set mener vi at alle invendingerne er ret svage. Vi valgte nedenstående indvendinger ikke fordi vi mener at de er særligt brugbare imod Turing-testen, men fordi det er de mindst svage af slagsen. De er altså valgt i mangel af bedre. % HAHAHAHA
\subsection{Lady Lovelace}
Argumentet går ud på, at en maskine kun kan det, vi beder dem om at gøre. Turing sammenligner det med spørgsmålet om, hvorvidt en maskine kan overraske os; han argumenterer blandt andet for, at maskiner ofte overrasker ved diverse komplicerede beregninger.

Der er generelt mange argumenter, der bygger på, at maskiner har mangler --- som man uden videre antager, mennesker ikke har. Det er selvsagt svage argumenter, idet man ikke argumenterer for sidstnævnte.

På trods af det gode modargument, er det stadig rimelig relevant. Computere nu til dags er så komplekse, at vi på ingen måde kan beregne det næste skridt. Derfor kan den overraske, eksempelvis ved AI i spil og emergent opførsel. Hvis man antager, at vores ideer og opfindelser kommer ned fra gud, så kan manskiner vel ikke det samme. Det giver mening på intuitivt niveau, at computere ikke kan overraske os og i sagens natur er deterministiske og forudsigelige.

\subsection{Nervesystemets kontinuitet}
%Hjernen er kontinuer. Maskinen er diskret. Men hvis maskinen klarer testen er det intet problem.
%Maskinnen vil ikke lide under at den ikke er kontinunær fordi den godt kan … interrogatoren (udsspørgeregenge) (eksaminator) fordi han ikke bruger sin  kontinuæritet. $\pi$. En kontinuær maskine har nogle muligheder som en diskret maskine ikke har.

Indvendingen berør det faktum at menneskets hjerne er kontiniuær i modsætning til en maskine, der med rimelighed kan modelleres som diskret. I denne sammenhæng mener vi det er relevant at se på to fortolkningner af indvendingen.

Den første fortolkning sår tvivl om hvorvidt en maskine vil være i stand til at klare turing-testen idet diskrete maskiner er skarpt mindre kapable end kontinuære maskiner og er derfor ikke i stand til at simulere disse fuldstændigt. Dette er dog ikke et reelt problem for hvis maskinen ikke klarer testen betyder det ikke at den ikke er tænkende så derfor vil skævvridninger til ulempe for maskinen ikke kompromittere testens integritet.

Den anden fortolkning består i at selv hvis maskinen klarer testen ændrer det ikke ved at maskinen ikke er kontinuær. Indvendingen er derfor at hvis en diskret maskine, der som nævnt er skarpt mindre kapabel end en kontinuær hjerne, kan bestå testen, er testen ikke i stand til at skelne testdeltagere af forskellige bevidsthedsniveauer (Bemærk: I denne fortolkning implicit påstås at en kontinuær maskine arbejder på et højere bevidsthedsplan end en diskret maskine).

Denne indvending er efter vores opfattelse væsentlig i den anden fortolkning fordi det er en reel anke imod testens validitet.

%I tilfælde af at maskinen klarer Turing-testen, vil dette betyde, at forskellen på hjernen og maskinen ikke er noget problem, da udspørgeren ikke vil komme til at se det kontinuerte --- han ser kun svarene, der er diskrete af natur.
%En kontinuer maskine vil dog have nogle muligheder, som en diskret maskine ikke har, men dette vil ikke kunne fremgå af testen.

\subsection{Blockhead-indvendingen}
%Du forstår det ikke selvom du har en opslagsliste. Ingen reflektion. 
%Er samtalen begrænset? Tabellem er ikke endelig. Tabel er ikke godt nok. Uendelig stor. Fordi det er et tabelopslag er det ikke symbol på rigtig intelligens. Imitere tænkning.
%Selvom man kan lave en maskine med en så stor tabel at den klarer turing-testen er det ikke udtryk for at maskinen kan tænke selv. Tabelopslag er ikke det samme som intelligens. Turing-testen beviser ingenting.
%Hvis man laver en maskine der "rigtigt" klarer turing-testen vil man også nemt kun skalere den op til 30 minutter.
%Selvom man klarer testen betyder det ikke at man er intelligent.

Blockhead-invendingen går grundlæggende ind og siger, at selvom du bruger det, behøver du ikke have en forståelse for det. Der mangler altså reflektion.
Hvis tabellen i Blockhead-invendingens eksempel er stor nok, og samtalen er begrænset nok, til at en maskine vil kunne besvare udspørgerens spørgsmål overbevisende nok til at bestå Turing-testen, medfører dette ikke at maskinen selv kan tænke eller er intelligent. Blockhead-invendingen siger altså ikke noget direkte om maskiner, men derimod noget om testen.

Maskinen kan enten være ``heldig'' eller god til specielle tilfælde, men hvis man skalerede testen op til at vare for eksempel ti minutter i stedet for fem, ville det ikke være sikkert, at maskinen stadig kunne bestå, men den ville stadig have klaret testen på de fem minutter, og testen er derfor ikke valid ifølge Blockhead-invendingen.

Vi mener Blockhead-indvendingen er væsentlig fordi den, ligesom indvendingen om nervesystemets kontinuitet, sætter spørgsmålstegn ved validitet af testen og fordi den som minimum rejser spørgsmålet om hvor lang en minimal Turing-test bør være. For en tilstrækkelig lang Turing-test vil lagerpladsen til tabellen dog være større end antallet af atomer i universet i hvilket fald metoden til at omgås Turing-testen i Blockhead-invendingen ikke længere er anvendelig.

\section{Træning af computere}
Kringelbach mener, at man er på tærsklen til at kunne simulere emotionel adfærd på et konceptuelt plan og dermed få computere, hvis adfærd er baseret på samme grundlag som den menneskelige hjerne. Til gengæld er der stadig principielle forskelle; for eksempel simuleres neuroner med envejssignaler i modsætning til rigtige neuroner, hvor signalerne går begge veje. Overordnet er der stadig lang vej igen, før realistiske simulationer er mulige --- noget så simpelt som en rundorm med 302 neuroner er ikke simulerbart i dag.

\end{document}
