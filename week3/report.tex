\documentclass{article}
\usepackage[utf8]{inputenc}
\usepackage{graphicx}
\usepackage{fancyhdr}
\usepackage{verbatim}
\usepackage[danish]{babel}
\pagestyle{fancyplain}
\author{Mikkel Kragh Mathiesen, Jannik Gram, Rune \& Rasmus Abrahams{\tt (son|en)}}
\title{}
\date{\today}
\lhead{Mikkel, Jannik, Rune \& Rasmus}
\rhead{\today}
\begin{document}
\maketitle
\section{Turings computermodel}
Kun digitale maskiner i testen fordi de kan bevises at være universelle turing-maskiner.
Altså kan en sådan digatal maskine simulere enhver discrete state maskine.
enhver digital maskine simulere enhver discrete state machine.

(han tillader digitale maskiner fordi de er begrænset til at simulere diskrete maskiner)

Forskellen: Digitale maskiner i teorien ikke kan være diskrete men i praksis (ikke er diskrete, eller ikke kan være, ikke er) de er ikke diskrete men kan i praksis ses som at være det. fordi man kan forudsige deres tilstand selvom en elektronspænding ikke er præcis så den regner rigtigt med det.
Den er i sig selv en diskret tilstandsmaskine men samtidig … ideen er at i en virkelig verden er en digital maskine ikke en  diskret tilstandsmaskine men det kan abstraheres væk…

\section{Computerhukommelsens betydning}
Nævnte $10^9$ (for en blind, ikke simulere syn)
Han gætter at hjernen har $10^{10}$-$10^{15}$ men hypotiserer at til turing-testen er $10^9$ nok.
Ideen er at hvis hjernenbruger $10^9$ så skal en simulering, der direkte simulerer hjernen bruge samme mængde hukommelse. Det er en måde at klare turing-testen på.
Neurale netværk
$10^9$ bytes? neuroner? Det er bits. (binary digits)
Han tror at det kun er (snarere er) $10^{10}$ end $10^{15}$.
Det meste af hvilket er brugt på processering af visuelle data.

\section{Jeres prioritering af indvendingerne mod Turing testen}
\subsection{Lady Lovelace}
Argumentet går ud på at en maskine kun kan det vi fortæller den skal gøre. Turing sammenligner det med spørgsmålet om en maskine kan overraske os. hvis man kigger på en stikkontakt. Den kan overraske fordi man selv sløser med beregninger.
Der er mange argumenter for at maskiner ikke kan et eller andet, mens man ignorerer at mnennesker ikke kan.
På trods af det gode modargument er det stadig rimeligt relevant lalalala.
Computere nu til dags er så komplekse at vi på ingen måde kan beregne det næste skridt. Derfor kan den overraske. F.eks. AI i spil.
Hvis man antager at vores ideer og opfindelser kommer ned fra gud huuuh så kan manskiner vel ikke det samme.
Det giver mening på intuitivt niveau.
Uendeligt mange aber.
Turings refusal. Jeg bliver sadigt overrasket af computere, selv hvis de opfører sig efter regler. Den er heldig at det lige passer at den overrasker en person.
Man kan beregne en computers næste skridt men ikke på mennesker.

\subsection{Nervesystemets kontinuitet}
Hjernen er kontinuer. Maskinen er diskret. Men hvis maskinen klarer testen er det intet problem.
Maskinnen vil ikke lide under at den ikke er kontinunær fordi den godt kan … interrogatoren (udsspørgeregenge) (eksaminator) fordi han ikke bruger sin  kontinuæritet. $\pi$. En kontinuær maskine har nogle muligheder som en diskret maskine ikke har.

\subsection{Blockhead indvendingen}
Du forstår det ikke selvom du har en opslagsliste. Ingen reflektion. 
Er samtalen begrænset? Tabellem er ikke endelig. Tabel er ikke godt nok. Uendelig stor. Fordi det er et tabelopslag er det ikke symbol på rigtig intelligens. Imitere tænkning.
Selvom man kan lave en maskine med en så stor tabel at den klarer turing-testen er det ikke udtryk for at maskinen kan tænke selv. Tabelopslag er ikke det samme som intelligens. Turing-testen beviser ingenting.
Hvis man laver en maskine der "rigtigt" klarer turing-testen vil man også nemt kun skalere den op til 30 minutter.
Selvom man klarer testen betyder det ikke at man er intelligent.


\section{Træning af computere}
Han mener at man er lige på tærsklen til at kunne simulere konceptet med emotionel adfærd men er langt fra overordnet at kunne simulere den menneskelige hjerne.
Selv adfærden for meget simple organismer som rundorme med 302 neuroner er stadig ikke succesfuldt blevet emuleret i alle aspekter
I simulerede neuroner går der kun signaler en vej, men i hjernen går de begge veje. Manglende kompleksitet.


\end{document}